
%%%%%%%%%%%%%%%%%%%%%%%%%%%%%%%%%%%%%%%%%%%%%%%%%

% UNIVERSIDADE FEDERAL DO PARANÁ (UFPR)
% SETOR DE CIÊNCIAS SOCIAIS APLICADAS
% PÓS-GRADUAÇÃO EM DESENVOLVIMENTO ECONÔMICO (PPGDE)
% DISCENTE: FELIPE DUPLAT LUZ

%%%%%%%%%%%%%%%%%%%%%%%%%%%%%%%%%%%%%%%%%%%%%%%%%

%%%%%% TRABALHO DE CONCLUSÃO DE CURSO (MONOGRAFIA, DISSERTAÇÃO OU TESE) %%%%%%

%----------------------------------------------------------------
%% Classe abntex2.cls:
%% abntex2.cls, v-1.9.5 laurocesar
%% Copyright 2012-2015 by abnTeX2 group at https://www.abntex.net.br/ 
%%
%----------------------------------------------------------------

% Classe:
\documentclass[
12pt,	        	% tamanho da fonte.
openright,	        % capítulos começam em pág ímpar (insere página vazia caso preciso).
%twoside,           % para impressão em anverso (frente) e verso.
oneside,            % para impressão em páginas separadas (somente anverso).
a4paper,			% tamanho do papel.
chapter=TITLE,		% títulos de capítulos convertidos em letras maiúsculas.
english,			% idioma adicional para hifenização.
french,				% idioma adicional para hifenização.
spanish,			% idioma adicional para hifenização.
brazil				% o último idioma é o principal do documento.
]{Classe/Classe_1}  % Classe_2 altera o cabeçalho.


% Pacotes utilizados:
\usepackage[T1]{fontenc}		  % seleção de códigos de fonte.
\usepackage[utf8]{inputenc}		  % codificação do documento (conversão automática dos acentos).
%\usepackage{lmodern}			  % usa a fonte Latin Modern.
\usepackage{times}		    	  % usa a fonte Times New Roman % para usar a fonte, lembre-se de tirar o % do comando %"\renewcommand{\ABNTEXchapterfont}{\rmfamily}".	
\usepackage{lastpage}			  % usado pela Ficha catalográfica.
\usepackage{indentfirst}		  % identa o primeiro parágrafo de cada seção.
\usepackage{color}				  % controle das cores.
\usepackage{graphicx}			  % inclusão de gráficos.
\usepackage{float} 				  % fixa tabelas e figuras no local exato.
\usepackage{chemfig}              % para escrever reações químicas.
\usepackage{chemmacros}           % para escrever reações químicas.
\usepackage{tikz}				  % para escrever reações químicas e outros.
\usetikzlibrary{positioning}      % posicionar imagens e tabelas.
\usepackage{microtype}            % para melhorias de justificação.
\usepackage{pdfpages}             % facilita alguma porra do pdf.
\usepackage{makeidx}              % para gerar índice remissivo.
\usepackage{hyphenat}             % retirar a hifenizacao do texto.
\usepackage[absolute]{textpos}    % permite o posicionamento do texto.
\usepackage{eso-pic}              % incluir imagem de fundo.
\usepackage{makebox}              % criar caixa de texto.
\usepackage{multicol}             % suporte a mesclagens em colunas.
\usepackage{multirow}	          % suporte a mesclagens em linhas.
\usepackage{longtable}	          % tabelas com várias páginas.
\usepackage{threeparttablex}      % notas no longtable.
\usepackage{array}                % formato de colunas.
%\usepackage[superscript]{cite}   % agrupar a citação numérica consecutiva.


% Sistema autor-data: (escolha apenas um!)
\usepackage[alf, abnt-emphasize=bf, abnt-thesis-year=both, abnt-repeated-author-omit=no, abnt-last-names=abnt, abnt-etal-cite, abnt-etal-list=3, abnt-etal-text=it, abnt-and-type=e, abnt-doi=doi, abnt-url-package=none, abnt-verbatim-entry=no]{abntex2cite}
\bibliographystyle{Classe/autordata}
%\bibliographystyle{Classe/autordata_ing} % para em inglês.


% Sistema autor-data com todos os autores:
%\usepackage[alf, abnt-emphasize=bf, abnt-thesis-year=both, abnt-repeated-author-omit=no, abnt-last-names=abnt, abnt-etal-cite, abnt-etal-list=0, abnt-etal-text=it, abnt-and-type=e, abnt-doi=doi, abnt-url-package=none, abnt-verbatim-entry=no]{abntex2cite} 
%\bibliographystyle{Classe/autordata}
%\bibliographystyle{Classe/autordata_ing} % para em inglês.


% Sistema numérico:
%\usepackage{cite}   % agrupa citações numéricas consecutivas.
%\usepackage[num, abnt-emphasize=bf, abnt-thesis-year=both, abnt-repeated-author-omit=no, abnt-last-names=abnt, abnt-etal-cite, abnt-etal-list=3, abnt-etal-text=it, abnt-and-type=e, abnt-doi=doi, abnt-url-package=none, abnt-verbatim-entry=no]{abntex2cite} 
%\bibliographystyle{Classe/citnum}
%\bibliographystyle{Classe/citnum_ing}


% No sistema numérico, habilite um dos comandos abaixo:
%\renewcommand{\thefootnote}{\fnsymbol{footnote}}  % inserção de símbolos em nota de rodapé.
%\renewcommand{\thefootnote}{\alph{footnote}}      % inserção de letras minúscula em nota de rodapé.
%\renewcommand{\thefootnote}{\Alph{footnote}}      % inserção de letras maiúscula em nota de rodapé.
%\renewcommand{\thefootnote}{\roman{footnote}}     % inserção de números romanos minúsculos  em nota de rodapé.
%\renewcommand{\thefootnote}{\Roman{footnote}}     % inserção de números romanos minúsculos  em nota de rodapé.


% Outros:
\renewcommand{\footnotesize}{\small} % diminuir a fonte das notas de rodapé.
\renewcommand{\ABNTEXchapterfont}{\rmfamily} % para utilizar a fonte Times New Roman.
\usepackage{Classe/ABNT6023-2018} % compatibilização com a ABNT NBR 6023:2018.
%\DeclareFieldFormat{url}{\bibstring{urlfrom}\addcolon\addspace\url{#1}} % tirar <> da URL.



% ----------------------------------------------------------
% DADOS INICIAIS
% ----------------------------------------------------------

% Escolher o tipo de TCC:
%\curso{grad} % graduação;
\curso{mest}  % se mestrado; ou
%\curso{doc}  % se doutorado.


% informações do PDF:
\makeatletter
\hypersetup{
	%pagebackref=true,
	pdftitle={\@title}, 
	pdfauthor={\@author},
	pdfsubject={\imprimirpreambulo},
	pdfcreator={LaTeX with abnTeX2},
	pdfkeywords={abnt}{latex}{abntex}{USPSC}{trabalho acadêmico},
	colorlinks=true,      % false: boxed links; true: colored links.
	%linkcolor=black,     % cor dos links: preto.
	%citecolor=black,     % cor dos links: preto.
	%filecolor=black,     % cor dos links: preto.
	%urlcolor=black,      % cor dos links: preto.
	linkcolor=blue,       % cor dos links: azul.
	citecolor=blue,       % cor dos links: azul.
	filecolor=magenta,    % cor dos links: azul.
	urlcolor=blue,        % cor dos links: azul.
	bookmarksdepth=4	  % 
}
\makeatother
 

%% Espaçamentos entre linhas e parágrafos:

\setlength{\parindent}{1.3cm} % tamanho do parágrafo.
\setlength{\parskip}{0.2cm}   % espaçamento entre um parágrafo e outro.
\makeindex                    % compila o sumário e índice.
\frenchspacing                % retira espaço extra obsoleto entre as frases.



%%%%%%%%%%%%%%%%%%%%%%%%%%%%%%%%%%%%%%%%%%%%%%%%%%%%%%%%%%%%%%%%%%%%%%%%%%%%%%%


% TEXTO COMEÇA A PARTIR DAQUI:


%%%%%%%%%%%%%%%%%%%%%%%%%%%%%%%%%%%%%%%%%%%%%%%%%%%%%%%%%%%%%%%%%%%%%%%%%%%%%%%

\begin{document}

% Idioma do documento:
\selectlanguage{brazil}   % se português.
%\selectlanguage{english} % se inglês.


% Formatação dos títulos:
\renewcommand{\ABNTEXchapterfontsize}{\fontsize{12}{12}\bfseries}
\renewcommand{\ABNTEXsectionfontsize}{\fontsize{12}{12}\bfseries}
\renewcommand{\ABNTEXsubsectionfontsize}{\fontsize{12}{12}\normalfont}
\renewcommand{\ABNTEXsubsubsectionfontsize}{\fontsize{12}{12}\normalfont}
\renewcommand{\ABNTEXsubsubsubsectionfontsize}{\fontsize{12}{12}\normalfont}



% ----------------------------------------------------------
% ELEMENTOS PRÉ-TEXTUAIS
% ----------------------------------------------------------

\imprimircapa                           % capa.
\imprimirfolhaderosto                   % inserir * se for twoside.

%---%


%---------------------------------------------------------------------
% Ficha catalográfica
%---------------------------------------------------------------------

\begin{fichacatalografica}
	\hspace{-1.4cm}
	\imprimirnotaautorizacao \\ \\
	%\sffamily
	\vspace*{\fill}				% posição vertical
\begin{center}					% minipage Centralizado
  \imprimirnotabib
  \begin{table}[htb]
	\scriptsize
	\centering	
	\begin{tabular}{|p{0.9cm} p{8.7cm}|}
		\hline
	      & \\
		  &	  \imprimirautorficha     \\
		
		 \imprimircutter & 
							\hspace{0.4cm}\imprimirtitulo~  / ~\imprimirautor~ ;  ~\imprimirorientadorcorpoficha. -- 	\imprimirlocal, \imprimirdata.   \\
		
		  &   %\hspace{0.4cm} \imprimirtitulo~  / ~\imprimirautor~ ; ~\imprimirorientadorcorpoficha~- ~\imprimirnotafolharosto. -- \imprimirlocal, \imprimirdata.  \\
		  
		  \hspace{0.4cm}\pageref{LastPage} p. : il. (algumas color.) ; 30 cm.\\ 
		  & \\
		  & 
		    \hspace{0.4cm}\imprimirnotaficha ~--~ 
						  \imprimirunidademin, 
						  \imprimiruniversidademin, 
		                  \imprimirdata. \\ 
		  & \\                 
		    % Para incluir nota referente à versão corrigida em notas,
		    % incluir uma % no início da linha acima e	
		    % tirar a % do início da linha abaixo
		    % & \hspace{0.4cm}\imprimirnotafolharosto \\ 
		  & \\ 
		  & \hspace{0.4cm}1. Palavra-chave. 2. Palavra-chave. 3. Palavra-chave. 4. Palavra-chave. 5. Palavra-chave. I. \imprimirorientadorficha. 
		   %II. Título. \\ % se não tiver co-orientador
		   III. Título da dissertação. \\ % se tiver co-orientador
		  \hline
	\end{tabular}
  \end{table}
\end{center}
\end{fichacatalografica}


       % ficha catalográfica.

%---------------------------------------------------------------------
% FOLHA DE APROVA��O
%---------------------------------------------------------------------

\begin{folhadeaprovacao}
  \begin{center}
       {\ABNTEXchapterfont\bfseries\large\imprimirautor}
	 \vspace*{2cm}
   
    \begin{center}
      \ABNTEXchapterfont\bfseries\Large\imprimirtitulo
    \end{center}
		\vspace*{1cm}
		\hspace{.45\textwidth}
    \begin{minipage}{.5\textwidth}
        \imprimirpreambulo
    \end{minipage}
		\vspace*{1cm}
    %\vspace*{\fill}
	\end{center}

  \begin{center}
	  {\ABNTEXchapterfont\bfseries\large\ {Data de defesa: xx de xxxx de 2024} \\}
		\vspace*{\fill}
	  {\ABNTEXchapterfont\bfseries\large\ {Comiss\~ao Julgadora:} \\}
		
		\assinatura{\textbf{\imprimirorientador} \\ Orientador} 
		
		\assinatura{\textbf{K\^enia Barreiro de Souza} \\ Co-orientadora}
	
		\assinatura{\textbf{Professor(a) XXXX} \\ Convidado}
		
		\assinatura{\textbf{Professor(a) XXXX} \\ Convidado}
		
		\vspace{0.5cm}
		
   	{\ABNTEXchapterfont\bfseries\large\imprimirlocal}
    \par
    {\ABNTEXchapterfont\bfseries\large\imprimirdata}
\end{center}
\end{folhadeaprovacao}


        % folha de aprovação.

%---------------------------------------------------------------------
% DEDICATÓRIA
%---------------------------------------------------------------------

\begin{dedicatoria}
   \vspace*{\fill}
   \centering
   \noindent
   
   Texto de dedicatória. 
   
   \vspace*{\fill}
\end{dedicatoria}


               % dedicatória.

%---------------------------------------------------------------------
% AGRADECIMENTOS
%---------------------------------------------------------------------

\begin{agradecimentos}
	
	Texto.
	
\end{agradecimentos}


            % agradecimentos.

%---------------------------------------------------------------------
% EPÍGRAFE
%---------------------------------------------------------------------

\begin{epigrafe}
    \vspace*{\fill}
	\begin{flushright}
		\textit{``Citação.''}
	\end{flushright}
\end{epigrafe}


                  % epígrafe.

%---------------------------------------------------------------------
% RESUMO
%---------------------------------------------------------------------

\setlength{\absparsep}{18pt} % ajusta o espaçamento dos parágrafos do resumo		
\begin{resumo}
	
Texto.

 \textbf{Palavras-chave}: Tese. Dissertação. Etc. 
\end{resumo}                    % resumo.

%---------------------------------------------------------------------
% ABSTRACT
%---------------------------------------------------------------------

\begin{resumo}[Abstract]
 \begin{otherlanguage*}{english}

   Text.

   \noindent 
   \textbf{Keywords}: Thesis. Dissertation. Etc.
    
 \end{otherlanguage*}
\end{resumo}
                  % abstract (vem antes se trabalho for em inglês).

%---%

\pdfbookmark[0]{\listfigurename}{lof}   % lista de figuras.
\listoffigures*                         % lista de figuras.
\cleardoublepage                        % lista de figuras.
\pdfbookmark[0]{\listtablename}{lot}    % lista de tabelas.
\listoftables*                          % lista de tabelas.
\cleardoublepage                        % lista de tabelas.
\pdfbookmark[0]{\listofquadroname}{loq} % lista de quadros.
\listofquadro*
\cleardoublepage

%---%


%---------------------------------------------------------------------
% ABREVIATURAS
%---------------------------------------------------------------------

\begin{siglas}
    
    \item[ECV] Esporte Clube Vitória.

	\item[]     

\end{siglas}
              % abreviaturas.

%---------------------------------------------------------------------
% SÍMBOLOS
%---------------------------------------------------------------------

\begin{simbolos}
  \item[$\alpha$] Letra grega Alfa em minúsculo.
  \item[$\beta$]  Letra grega Beta em minúsculo.
  \item[$\gamma$] Letra grega gama em minúsculo.
\end{simbolos}


                  % símbolos.

%---%

\pdfbookmark[0]{\contentsname}{toc}     % sumário.
\tableofcontents*                       % sumário.
\cleardoublepage                        % sumário.



% ----------------------------------------------------------
% ELEMENTOS TEXTUAIS
% ----------------------------------------------------------

\textual

% ----------------------------------------------------------
% CAPÍTULO 01 - INTRODUÇÃO
% ----------------------------------------------------------

\chapter{Introdução}

Texto. \cite{winters02, bannisterthugge01, castilho12}.

% ----------------------------------------------------------
% CAPÍTULO 02 - Revisão de literatura
% ----------------------------------------------------------

\chapter{Revisão de literatura}

Texto.

% ----------------------------------------------------------
% CAPÍTULO 03 - METODOLOGIA E DADOS
% ----------------------------------------------------------

\chapter{Metodologia e dados}

Texto.


	


% ----------------------------------------------------------
% CAPÍTULO 04 - RESULTADOS
% ----------------------------------------------------------

\chapter{Resultados}

Texto.


	


% ----------------------------------------------------------
% CAPÍTULO 05 - CONSIDERAÇÕES FINAIS
% ----------------------------------------------------------

\chapter{Considerações finais}

Texto.

	




% ----------------------------------------------------------
% ELEMENTOS PÓS-TEXTUAIS
% ----------------------------------------------------------

\postextual
\bibliography{Referências}

%---------------------------------------------------------------------
% APÊNDICES
%---------------------------------------------------------------------

\begin{apendicesenv}

\chapter{título}

Texto.

\end{apendicesenv}



%---------------------------------------------------------------------
% ANEXOS
%---------------------------------------------------------------------

\begin{anexosenv}

\chapter{título}

Texto.

\end{anexosenv}


\end{document}


